
    \documentclass{article}
    \usepackage[legalpaper, left=1 cm, right=1cm, top=0.5cm, bottom=0.5cm] {geometry}
    \date{} % Remove a exibição da data
    \usepackage{xcolor}
    \usepackage{listings}
    \usepackage{graphicx}
    \usepackage{hyperref} % Para criar links
    \usepackage[utf8]{inputenc}
    \usepackage[T1]{fontenc}

    \lstset{
    language=python,
    basicstyle=\ttfamily,
    keywordstyle=\bfseries\color{blue},
    commentstyle=\color{blue},
    stringstyle=\color{red!70!black},
    numberstyle=\tiny,
    stepnumber=1,
    numbersep=5pt,
    backgroundcolor=\color{white},
    breaklines=true,
    breakautoindent=true,
    showspaces=false,
    showstringspaces=false,
    showtabs=false,
    tabsize=2,
    literate={~}{{\textasciitilde}}1, % Trata ~ como um caractere normal
    extendedchars=true, % Permite caracteres estendidos (acentos, etc.)
    inputencoding=utf8, % Define a codificação de entrada como UTF-8
    literate={á}{{\'a}}1 {ã}{{\~a}}1 {ç}{{\c{c}}}1
    }
    \title{Docker}
    \begin{document}
    \maketitle
    \section{Converter aplicação para imagem}
        \subsection{Dockerfile}
            \begin{lstlisting}
                FROM <plataforma>:<sistema_operacional>
                COPY <origem> / < destino>
                WORKDIR <diretorio_trabalho>
                CMD <comando_run>
            \end{lstlisting}
            \subsubsection{Construindo imagem}
                \begin{lstlisting}
                    #construindo diretorio local
                    docker build -t <nome_imagem>
                    #vizualizando as imagens
                    docker images 
                \end{lstlisting}

    \section{Application}
        \subsection{Adicionando sistemas operacionais}
            \begin{lstlisting}
                FROM node:12- alpine #adiciona alinguagem e o sistema operacional 
                WORKDIR /App #Especificao direroiro de trabalho
                COPY .. # Copia todo diretorio
            \end{lstlisting}
        \subsection{Comando RUN}
            \begin{lstlisting}
                FROM node:12-alpine #image do docker hub

                WORKDIR /app # diretorio do image

                COPY . . # copia dos arquivos

                ADD https://microsoft.com/teste.json .% copia dos arquivos online compactados ou descompactados
                RUN apk add --no-cache python2 g++ make % adiconando python no alpine linux
                CMD [ ]
            \end{lstlisting}    
            \textcolor{red}{RUN apk add --no-cache python2 g++ make}
            \begin{itemize}
                \item apk: é o utilitário de gerenciamento de pacotes do Alpine Linux.
                \item add: é o subcomando utilizado para adicionar pacotes.
                \item --no-cache: é uma opção que instrui o apk a não salvar o índice localmente após a instalação, o que economiza espaço em disco, mas também significa que não será possível atualizar ou reinstalar pacotes sem uma conexão com a internet.
                \item python2: é o pacote do Python 2 que será instalado.
                \item g++: é o compilador C++.
                \item make: é uma ferramenta de automação de compilação utilizada principalmente para compilar e construir projetos de software
            \end{itemize}
        
    

    \end{document}
    \section{}
    \begin{lstlisting}
    \end{lstlisting}

